\documentclass{article}
\usepackage[T1]{fontenc}
\usepackage[utf8]{inputenc}
\usepackage[portuguese]{babel}

\title{Relatório EP1}
\author{Gabriel Henrique Pinheiro Rodrigues NUSP: 112.216-47}
\date{Março 2020}

\begin{document}
\maketitle

\part*{Implementação}   
As tabelas de símbolos foram implementadas seguindo o padrão <Chave, Valor> sendo Chave do tipo 
string e Valor do tipo int, portanto quando uma chave ou valor não é encontrado na tabela, é retornado
respectivamente $``$ '' e -1;

\section{Vetor Desordenado}

\subsection{Insere $O(n)$} 
É necessário verificar se o elemento a ser inserido pertence a tabela ou não, se não pertence insere no fim.
\subsection{Devolve $O(n)$}
Não há relação de ordem, logo é necessário checar todos os elementos da tabela até encontrar.
\subsection{Remove $O(n)$}
Não foi utilizado lazy deletion, logo é necessário buscar o elemento e redimensionar o array.
\subsection{Rank $O(n)$}
Percorre toda a tabela, comparando a chave dos elementos com a chave analisada.
\subsection{Seleciona $O(n^2)$}
Percorre toda a tabela, chamando a função rank para cada elemento até encontrar um com rank igual a k.

\section{Vetor Ordenado}

\subsection{Insere $O(n)$} 
É necessário verificar se o elemento a ser inserido pertence a tabela ou não, se não pertence insere 
e desloca todos os elementos maiores do que ele.
\subsection{Devolve $O(\log_{2}n)$}
Há relação de ordem, logo é possível fazer busca binária.
\subsection{Remove $O(n)$}
Não foi utilizado lazy deletion, logo é necessário buscar o elemento ($\log_{2}n$) e redimensionar o array (Pior 
caso: O(n)).
\subsection{Rank $O(\log_{2}n)$}
Como há relação de ordem entre os elementos, basta procurar o primeiro elemento maior do que o elemento da 
chave.
\subsection{Seleciona $O(1)$}
Como não foi utilizado lazy deletion o elemento de rank k é o elemento de índice k.

\section{Lista Ligada Desordenada}

\subsection{Insere $O(n)$} 
É necessário verificar se o elemento a ser inserido pertence a tabela ou não, se não pertence insere no início.
\subsection{Devolve $O(n)$}
Não há relação de ordem, logo é necessário checar todos os elementos da tabela até encontrar.
\subsection{Remove $O(n)$}
A eliminação do elemento é $O(1)$, porém a busca do elemento é $O(n)$.
\subsection{Rank $O(n)$}
Percorre toda a tabela, comparando a chave dos elementos com a chave analisada.
\subsection{Seleciona $O(n^2)$}
Percorre toda a tabela, chamando a função rank para cada elemento até encontrar um com rank igual a k.

\section{Lista Ligada Ordenada}

\subsection{Insere $O(n)$} 
É necessário verificar se o elemento a ser inserido pertence a tabela ou não, se não pertence insere de maneira
ordenada.
\subsection{Devolve $O(n)$}
Há relação de ordem, mas não há acesso imediato a qualquer elemento da tabela, logo é necessário avançar até
o elemento com a chave correspondente.
\subsection{Remove $O(n)$}
A eliminação do elemento é $O(1)$, porém a busca do elemento é $O(n)$.
\subsection{Rank $O(n)$}
Percorre a tabela enquanto as chaves são menores do que a chave analisada.
\subsection{Seleciona $O(n)$}
Percorre a tabela até chegar no elemento de índice k.
\end{document}